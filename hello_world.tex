\input cwebmac

\N{1}{1}An example of {\tt CWEB}.
This example is blah blah blah.
I just hope it works.

\fi

\M{2}The basic file structure for a {\tt C} program is as follows:

\Y\B\X3:Include section\X\X4:File specific prototypes and global variables\X%
\X5:Main function\X\X6:Function implementations\X\par
\fi

\M{3}This is the \TeX part.
It contains explanatory material about what is going on in the section.
The include section is where header files are included to provide prototypes
for library functions.
If \PB{\\{pa}} is declared as `\PB{\&{int} ${}{*}\\{pa}$}`, the assignment %
\PB{$\\{pa}\K{\AND}\|a[\T{0}]$} makes \PB{\\{pa}} point to the zeroth element
of \PB{\|a}.

\Y\B\4\X3:Include section\X${}\E{}$\6
\8\#\&{include} \.{<stdio.h>}\par
\U2.\fi

\M{4}If necessary, this section is typically where file specific function
prototypes and global variables are defined.

\Y\B\4\X4:File specific prototypes and global variables\X${}\E{}$\6
\&{static} \&{const} \&{char} ${}{*}\\{msg}\K\.{"Hello\ world!\ This\ i}\)\.{s\
the\ shit"};{}$\6
\&{void} \\{say\_hello\_world}(\,);\par
\U2.\fi

\M{5}This is where the magic happens.
The \PB{\\{main}} function is what gets executed when the program is run.
By convention, the \PB{\\{main}} function returns an \PB{\&{int}}.
The \PB{\&{int}} is generally an exit code.

\Y\B\4\X5:Main function\X${}\E{}$\6
\&{int} \\{main}(\,)\1\1\2\2\6
${}\{{}$\1\6
\\{say\_hello\_world}(\,);\6
\&{return} \T{0};\6
\4${}\}{}$\2\par
\U2.\fi

\M{6}If applicable, this is generally where file specific function
implementations go.
In the case of the hello world example, the say hello world function needs an
implementation.

\Y\B\4\X6:Function implementations\X${}\E{}$\6
\&{void} \\{say\_hello\_world}(\,)\1\1\2\2\6
${}\{{}$\1\6
\\{puts}(\\{msg});\6
\4${}\}{}$\2\par
\U2.\fi

\inx
\fin
\con
