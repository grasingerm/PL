\input cwebmac

\N{1}{1}An example of {\tt CWEB}.
This example is blah blah blah.
I just hope it works.

\fi

\M{2}This is the \TeX part.
It contains explanatory material about what is going on in the section.
The include section is where header files are included to provide prototypes
for library functions.
If \PB{\\{pa}} is declared as `\PB{\&{int} ${}{*}\\{pa}$}`, the assignment %
\PB{$\\{pa}\K{\AND}\|a[\T{0}]$} makes \PB{\\{pa}} point to the zeroth element
of \PB{\|a}.

\Y\B\8\#\&{include} \.{<stdio.h>}\par
\fi

\M{3}If necessary, this section is typically where file specific function
prototypes and global variables are defined.

\Y\B\&{static} \&{const} \&{char} ${}{*}\\{msg}\K\.{"Hello\ world!\ This\ i}\)%
\.{s\ the\ shit"};{}$\6
\&{void} \\{say\_hello\_world}(\,);\par
\fi

\M{4}This is where the magic happens.
The \PB{\\{main}} function is what gets executed when the program is run.
By convention, the \PB{\\{main}} function returns an \PB{\&{int}}.
The \PB{\&{int}} is generally an exit code.

\Y\B\&{int} \\{main}(\,)\1\1\2\2\6
${}\{{}$\1\6
\\{say\_hello\_world}(\,);\6
\&{return} \T{0};\6
\4${}\}{}$\2\par
\fi

\M{5}If applicable, this is generally where file specific function
implementations go.
In the case of the hello world example, the say hello world function needs an
implementation.

\Y\B\&{void} \\{say\_hello\_world}(\,)\1\1\2\2\6
${}\{{}$\1\6
\\{puts}(\\{msg});\6
\4${}\}{}$\2\par
\fi

\inx
\fin
\con
